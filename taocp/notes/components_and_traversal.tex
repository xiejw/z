\section{Components and Traversal}

\urldef{\wikipediaStrongConnectedComponents}\url{https://en.wikipedia.org/wiki/Tarjan%27s_strongly_connected_components_algorithm}
\urldef{\codeAlgoTStrongConnectedComponents}\url{https://github.com/xiejw/z/blob/7dcea1ad9d450bff76aa87ec102bf6d7140a1134/taocp/v4_7.4.1.2.algo_t_strong_components/src/graph_sgb.cc#L15-L214}

There is a distinct beauty in Algorithm 7.4.1.2T; it avoids the heavy-handedness
of global scans in favor of a singular, elegant pass. By allowing the
\texttt{LOW} value to cascade through the graph's hierarchy, the algorithm
listens to the local state of each vertex to pinpoint the exact moment a
component is born. It exemplifies the brilliance of an algorithm that makes
`smart' decisions not through complexity, but through the perfect placement of
simple information.

\paragraph{Iterative DFS}
Upon closer inspection, Algorithm T follows an iterative Depth-First Search
(DFS) structure characterized by precise stack management and explicit control
flow jumps. This design ensures that all memory allocations, stores, and loads
remain transparent. However, these advantages come with a trade-off: the
algorithm requires rigorous engineering to maintain critical invariants. I have
documented these invariants extensively within the codebase\sidenote{See code at
\codeAlgoTStrongConnectedComponents.} to ensure clarity and maintainability.

\paragraph{Recursive DFS}
In contrast, it is more common to conceptualize and implement this process using
recursive function calls, as seen in the standard Wikipedia
implementation\sidenote{See psedu code at \wikipediaStrongConnectedComponents.}.
However, this approach carries the risk of stack overflow when processing large
graphs and tends to obscure the inherent overhead of frequent function calls.

