% VERSION: 2024-12-29

% === --- Import the basic template ---------------------------------------- ===
% See: https://tug.org/TUGboat/tb34-1/tb106olsak-opmac.pdf
\input opmac

% === --- Load fonts ------------------------------------------------------- ===
\font\ninett=cmtt9
\font\secfont=cmss12 %14 seems broken
\font\subsecfont=cmss10

% === --- Define colors ---------------------------------------------------- ===
% See http://www.december.com/html/spec/cmykshades.html
\def\StarbuckGreen{\setcmykcolor{1.0 0 0.5 0.6}}

% === --- Margin note ------------------------------------------------------ ===
\fixmnotes\right \mnotesize=1.7in \def\mnotehook{\typosize[8/10]\it}
\mnoteindent=20pt
\margins/1 letter (1,2.5,,)in

% === --- Hyper Link ------------------------------------------------------- ===
\hyperlinks \Green \Blue % Internal green, external blue.

% === --- Adjust verbatim size --------------------------------------------- ===
\def\tthook{\typosize[9/11]}  % Adjust size for verbatim.

% === --- Define \symbol macro --------------------------------------------- ===
%
% * It starts with a group and redfines the undercore (_) to be normal char
% * As all arguments are read when process macro. The macro should be defined as
%   two macros, so the underscore in the argument can be handled correctly.
% * \relax makes the \catcode effective immediately. A {} (empty group) or empty
%   space can work also. This is the space-after-number rule (See TexBook Page
%   208).
% * In the second macro, we use the OPmac local color to change the group as
%   green color.
\def\symbol{\begingroup\catcode`\_=12\relax\symbolimpl}
\def\symbolimpl#1{{\ninett #1}\endgroup}

% === --- Define \sec and \subsec macros ----------------------------------- ===
%
% This is very similar to the `\date` macro in the format.tex file.
% - Insert a title in bold.
% - Skip 5pt vertically.
% - The first paragraph has no indent. This is done via the \everypar, which is
%   invoked at the start of each paragraph. It sets an empty box at first and
%   resets itself.
% - [New change] A black triangle is inserted left to the margin.
\def\sec#1{\vskip15pt
\noindent{\llap{\raise0.95pt\hbox{$\blacktriangleright$ }}\secfont{}#1}\hfill\vskip 5pt
  \everypar={{\setbox0\lastbox}\everypar={}}}

\def\subsec#1{\vskip10pt \noindent{\underbar{\subsecfont{}#1}}\hfill\vskip 5pt
  \everypar={{\setbox0\lastbox}\everypar={}}}

% ------------------------------------------------------------------------------
% \def\begtt{%
%   \begingroup
%     \parindent=0mm % remove indent
%     \def\do##1{\catcode`##1=12 }%
%     \dospecials
%     \obeylines
%     \obeyspaces
%     \tt
%     \verbatimaux
% }
% \begingroup
% \catcode`\|=0 %
% \catcode`\\=12 %
% |gdef|verbatimaux#1\endtt{#1|endtt}%
% |endgroup
% \def\endtt{\endgroup}
% {\obeyspaces\gdef {\ }}% Omit this for 'explicit' spaces
